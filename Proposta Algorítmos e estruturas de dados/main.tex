\documentclass[a4paper]{article}

%% Language and font encodings
\usepackage[english]{babel}
\usepackage[utf8x]{inputenc}
\usepackage[T1]{fontenc}
\usepackage{float}

\usepackage{mathtools}
\DeclarePairedDelimiter\ceil{\lceil}{\rceil}
\DeclarePairedDelimiter\floor{\lfloor}{\rfloor}

%% Sets page size and margins
\usepackage[a4paper,top=3cm,bottom=2cm,left=3cm,right=3cm,marginparwidth=1.75cm]{geometry}

%% Useful packages
\usepackage{amsmath}
\usepackage{graphicx}
\usepackage[colorinlistoftodos]{todonotes}
\usepackage[colorlinks=true, allcolors=blue]{hyperref}
\usepackage{multirow}

\title{Tema: Estruturas de dados aplicadas à arduino}
\author{Arleson Roberto - Matrícula: 5231  \\ Geidson Vinícius Gomes Barbosa - Matrícula: 6331 \\ Isaías Souza Silva - Matrícula: 4874 }

\date{06 de novembro de 2019}

\begin{document}
\maketitle

\section{Introdução}
Neste projeto iremos usar a estrutura de dados lista duplamente encadeada circular em conjunto com um arduino para a resolução do problema de baixo consumo de água durante o dia, que muitas vezes acontece devido ao esquecimento do usuário pois o mesmo se perde em suas atividades diárias, assim esse projeto estará auxiliando o usuário a ingerir a quantidade recomendada de água diariamente.

\section{Cenário de Aplicação}
Gleidson estava acima do peso e sofrendo com diversos problemas de saúde devido à sua condição. Após procurar ajuda médica e ouvir que a melhor situação ao longo do prazo para seus problemas seria uma reeducação alimentar combinada com uma rotina de exercícios físicos que resultaria em uma perda de peso, assim decidiu começar imediatamente sua jornada para uma vida mais saudável. Gleidson foi a um nutricionista e com sua nova dieta, percebeu que seu consumo diário de água estava muito abaixo do adequado para seu peso atual. Com sua rotina de trabalho exaustiva no escritório muitas vezes ele mergulhava em suas tarefas e passava horas a fio sem consumir água.\\Decidido a passar pelo processo de reeducação alimentar, ele então idealizou que iria usar seus conhecimentos de programação e um arduino para criar o que ele posterior mente chamou de "Stay Hydrated!", um programa que irá ajudá-lo a controlar seu consumo de água e lembrá-lo de ingerir água regularmente durante o dia. 


\section{Solução do Problema}
O uso da lista duplamente encadeada circular se encaixa perfeitamente nesse projeto pois, o fato de ser circular permite ao fim do dia voltar ao início e se repetir no novo dia apenas seguindo a lista, assim como o fato de poder retornar ao elemento anterior da lista auxilia com a confirmação do consumo de água, retornando à célula anterior caso não seja feita.


No primeiro acesso do usuário, o programa faz a única coleta de dados necessária para o cáculo: o peso do usuário. Ele então calcula a quantidade de água a ser consumida pelo usuário através de um cálculo predefinido e em seguida o programa calculará o tamanho do intervalo em que será lembrado repetidamente durante o dia, em intervalos iguais, para consumir água. A partir daí o programa dividirá a quantidade de água que deverá ter sido ingerida até o final do dia, levando em consideração que esse consumo não seja maior que 400ml de água, de modo que o usuário possa consumir uma quantidade exata de mililitros ao final de cada intervalo.


Finalizado o tempo do intervalo, o programa acenderá uma luz verde, exibirá uma mensagem “Hora de se hidratar!” no monitor e emitirá um som para avisar o usuário de que é hora de consumir água. Um botão deve ser pressionado como sinal de que o usuário está ciente da mensagem e irá tomar água e ser pressionado novamente para e a quantidade de água que será consumida, com as opções de consumir o equivalente a um ou dois intervalos. Após isto a luz se tornará azul, o monitor será desligado e o programa reiniciará a contagem para o próximo intervalo. Caso o botão não seja pressionado no tempo determinado a luz se tornará amarela e o consumo de água será adiado até o próximo intervalo.Ao fim do consumo diário o programa irá aguardar 12 horas para iniciar o novo dia e apresentará a opção de encerrar o programa ao usuário, caso este opte por encerrar a lista será destruída e o programa encerrado.



\end{document}